\documentclass[12pt]{article}
	\usepackage[T1]{fontenc}
	\usepackage[utf8]{inputenc}
	\usepackage[british]{babel}
	\usepackage[a4paper]{geometry}
	\geometry{verbose,tmargin=3cm,bmargin=3cm,lmargin=2cm,rmargin=2cm,marginparwidth=70pt}
	\setcounter{secnumdepth}{3}
	\setcounter{tocdepth}{3}
	\setlength{\parindent}{4em}
	\setlength{\parskip}{1em}
	\renewcommand{\baselinestretch}{1.5}
	\usepackage{prettyref}
	\usepackage{textcomp}
	\usepackage{setspace}
	\usepackage{indentfirst}
	\usepackage{fancyhdr}
	\usepackage{url}
	\usepackage[normalem]{ulem}
	\usepackage[table, fixpdftex]{xcolor}
	\usepackage{algpseudocode}
	\usepackage{bigstrut}
	\usepackage{enumitem}

	% package hyperref
    \usepackage[hidelinks]{hyperref}
  

	% biblatex
	\usepackage[style=authoryear,natbib=true,maxcitenames=2, maxbibnames=11,backend=biber,pagetracker=page,hyperref=true]{biblatex} \usepackage{csquotes}
	\renewcommand*{\bibsetup}{%
		\interlinepenalty=10000\relax % default is 5000
		\widowpenalty=10000\relax
		\clubpenalty=10000\relax
		\raggedbottom
		\frenchspacing
        \biburlsetup}
        
	% fixes the page number of the first page of each chapter
	\fancypagestyle{plain}{
			\fancyhead{}
			\renewcommand{\headrulewidth}{0pt}
			\renewcommand{\footrulewidth}{0pt}
			\fancyfoot[OC]{\begin{flushright}\thepage\end{flushright}}
    }
    
	% fancy headers for the thesis
	\fancyhead{}
	\fancyhead[RO]{\slshape \nouppercase \rightmark}
	\fancyfoot[OC]{\begin{flushright}\thepage\end{flushright}}
	\renewcommand{\headrulewidth}{0.4pt}
	\setlength{\headheight}{14pt}

	% add bibliography database
	\addbibresource{BA copy.bib}
	
	% space between biblio items
	\setlength\bibitemsep{1.7\itemsep} 
	
	% title without ""
	\DeclareFieldFormat[inbook]{title}{#1}
	% non-italic
	\DeclareFieldFormat[online]{tlaitle}{#1} 
	% title unquoted
	\DeclareFieldFormat[article]{title}{#1} 
	% no pp. 
	\DeclareFieldFormat[article]{pages}{#1} 
	% bold volume
	\DeclareFieldFormat*{volume}{\mkbibbold{#1}\setpunctfont{\textbf}}
	
	% no in:
	\renewbibmacro{in:}{} 
	
	% (volume)
	\renewbibmacro*{volume+number+eid}{%
			\printfield{volume}%
			%\setunit*{\adddot}% DELETED
			% \setunit*{\addnbspace}% NEW (optional); there's also \addnbthinspace
			\printfield{number}%
			% \setunit{\addcomma\space}%
			\printfield{eid}}
	\DeclareFieldFormat[article]{number}{\mkbibparens{#1}} 
	
	% edition.
	\DeclareFieldFormat{edition}%
	{(\ifinteger{#1}%
			{\mkbibordedition{#1}\addthinspace{}ed.}%
			{#1\isdot}).}
	
	% publisher and location position
	\renewbibmacro*{publisher+location+date}{%
			\printlist{publisher}%
			\setunit*{\addcomma\space}%
			\printlist{location}%
			\setunit*{\addcomma\space}%
			\usebibmacro{date}%
			\newunit}
	
	% shortauthor before author
	\renewbibmacro*{begentry}{%
			\ifkeyword{Key}{\sffamily}{}%
			\iffieldundef{shorthand}
			{}
			{\global\undef\bbx@lasthash
					\printfield{shorthand}%
					\addcolon\space}%
			\ifboolexpr{test {\usebibmacro{bbx:dashcheck}} or test {\ifnameundef{shortauthor}}}%
			{}%
			{\printnames{shortauthor}%
                    \addspace\textendash\space}}
                    
\title{Thesis}
\author{Leopold Ingenohl}


\begin{document}
\maketitle

\pagebreak


\section{Introduction}

\begin{center}
	Macht das überhaupt Sinn was ich schreibe? Kann man das nachvollziehen? 
\end{center}
% rethink sentence - not reporting requirements! 
Much attention has been recently given to the current Securities and Exchange Commission (SEC) reporting requirements for Schedule 13(D), governing the disclosure of beneficial ownership interests in excess of five percent of outstanding common stock of a U.S. public company \citep{Giglia2018}. Amongst other causes, it is due to significant gains for the subject's stock when a partial acquisition through a Schedule 13(D) filing is announced \citep{Akhigbe2007}. 

% types of filer - bridge corporations
However, it is still largely unanswered where this upward drift comes from \citep{Greenwood2009}. An approach to this issue is objective of this thesis. Namely, analyzing the link between the financial condition of corporate investors and the abnormal returns on the subject's stock and determining whether the financial condition has explanatory power for the latter. The following findings motivate this approach. 


In recent studies of what happens to the target's stock after such a filing, \citet{Collin-Dufresne2015} observe a positive significant market reaction to the subject's stock upon a more general sample of Schedule 13D filings
	\footnote{The sample is only restricted on the subjects stock characteristics rather than on characteristics of the filers e.g. they exclude all filings which are not common stock (CRSP share code 10 or 11), whose prices are below \$1 and above \$1000 and which involve derivatives \citep{Collin-Dufresne2015}.}. 
\citet{Brav2008} have shown a favorable market reaction, 7\% - 8\% average abnormal returns in the (-20|20) event window, particularly to Schedule 13D's filed by hedge funds. Similar results have been shown by \citet{Klein2009} who observe 10.2\% average abnormal stock returns specifically for hedge fund targets.\\
In addition, \citet{Brigida2012} have shown an even higher runup if the acquirer is a private investor or a non-financial corporation. This is matching with \citet{Akhigbe2007} findings who observe greater gains for the target's stock if the partial position was initiated by a corporate bidder. Concluding, filings submitted by all investor types are followed by positive market reactions on the subject's stock but those submitted by corporations seem to have a stronger impact. This motivates the first hypothesis which assumes significant positive abnormal returns for Schedule 13(D)'s filed by corporations.
% Characteristics | Motivation | Reasoning - Why do they make minority acquisitions? 

Since the investing corporation is allowed to behave in an activist manner by filing a Schedule 13(D) 
	\footnote{In comparison the investor could file a Schedule 13(G) in which he would hold the shares passively hence with no intention to bring change.} 
\citep{Brigida2012} they can use their stakes to actively monitor and influence the target which is similar to the definition of an entrepreneurial activist by \footnote{\citet{Klein2009} define the entrepreneurial activist as an investor who buys a large stake in a publicly held corporation with the intention to bring change and thereby realize a profit on the investment.} \citet{Klein2009}.
These stakes tend to be either made for the purpose of investment or far more importantly, as strategic investments \citep{Damodaran2005}, possibly resulting in business agreements, alliances or joint ventures \citep{Allen2000}. \\
% Topics: Acquisitions | Friendly Takeovers | Hostile Takeovers | Joint-Venture | Activism | Comparison | Comparison HF - Why do they behave activist? 
In a more direct approach however, these strategic investments can also help as a stepping stone towards full control \citep{Huang2017}. 
This approach is supported by \citet{Goldman2005} who find that mergers and takeovers are often preceded by the acquisition of a minority stake in the target. Whereas hedge funds use their stakes to change characteristics of the target (e.g. the board of directors or the strategic orientation) \citep{Klein2009} corporate filers are mainly focused on synergies in the form of strategic alliances or takeovers between them and the target. \citet{Akhigbe2007} observe that partial acquisitions, if carried out by corporate investors, are more likely to result in a full acquisition when compared to all other activist investors. This means that within the mass of Schedule 13D filings, institutional investors are unlikely to pursue a complete takeover whereas corporations are potential full acquirers \citep{Brigida2012}. The possibility of a takeover could be one explanation for the strong impact corporate filings have on the market, because the abnormal returns could be a reflection of investors' expectations of the target firms stock being acquired at a premium to the current price \citep{Goldman2005} especially with strong corporate bidders being likely to overpay in the event of a full takeover \citep{Akhigbe2007}.
These findings motivate the second hypotheses which assumes the highest abnormal returns occur in the event of a purpose of transaction statement involving a merger or a takeover of the subject.
%Objective: highlight the importance of the financial condition -- Background

However, in order to be able to bring change -- might it be in the form of a strategic alliance or eventually in a takeover -- the filing corporation should be in a condition of sufficient financial health. 
% insert the method of payments in a takeover 
A recent example on this matter is the public perception of the HNA Group. The financial condition of the HNA group, China's largest private conglomerate which over the past few years invested around \$US40 billion in businesses around the world, has currently been of great interest to financial news. Not least because they built up a 9.9\% stake of of around \$US4 billion in Deutsche Bank in 2017, which is just below the 10\% threshold above which stake purchases must be approved by Germany's financial watchdog but also because of their complex and nontransparent financing methods.
The financing of the group has come under strain as a result of an official crackdown on risky financing at acquisitive private enterprises in China. The highly leveraged group is now facing a potential cash-shortfall and liquidity issues resulting in a S\&P global rating downgrade referring to a a „deteriorating liquidity profile" of HNA. Although HNA group is a private conglomerate, the financial condition of corporations seems to be of great importance to other market participants with that said, even in the context of minority acquisitions. Therefore, linking investors' financial condition to  underlying market reactions could be an explanation for the latter. This motivates the third and most important hypotheses, namely that abnormal returns, triggered by activist minority acquisitions, can be explained by the financial condition of the investor. 
% payment method in takeovers! 

% Topics: Abnormal Returns | Financial Condition | Activism | Hypothesis | Conclusion

Based on the previous findings of corporate activism, namely their strong impact on the subjects stock in the form of abnormal returns and future possibilities involving the target, the economic significance of corporations as filers of Schedule 13(D)'s seems to be apparent.

Yet in order to make these possible developments and expectations look credible -- amongst other things strategic alliances and takeovers -- the investing corporation somehow has to emit signs of sufficient financial strength. Therefore, the link between the financial condition of the investor and the subsequent abnormal returns on the target's stock is an interesting issue to examine. This in particular, is objective of the paper. What precisely are the effects of Schedule 13(D) filings by corporations on the subject's stock and can the financial condition of the corporation explain the market's reaction? Or in other words -- how important is the financial condition of the corporation behaving in an activist manner? 

The paper proceeds as follows. In the Section 2, the relevant literature is being reviewed. Section 3 describes the data and sample composition. In Section 4 the market's response to Schedule 13(D) filings are being examined. Section 5 represents the ............
by  are being described.  In the section 4, being described 

\section{Hypotheses}

\begin{enumerate}
	\item There are significant positive abnormal returns after the Schedule 13(D) filing of a corporation
	\item The purpose of the transaction has an effect on the market reaction 
	\item The financial condition of the investor can explain the market reaction
	\item The financial condition is most important, when the puspose of transaction invovles a future merger or takeover
	\item The financial condition looses its importance when the target is a poorly performing company and gains importance when the target is performing well 
	\item 
\end{enumerate}

\section{Literature Review}

\subsection{Schedule 13(D) Filings}
%Topics: Historical Background | Information contained | Difference G and D
Section 13(d) of the Exchange Act of 1934 was passed in order to increase regulation of tender offers and accumulations of stock and the "growing use of cash tender offers as a means for achieving corporate takeovers."  
It acts as an early warning, signaling "every large, rapid aggregation or accumulation of securities, regardless of technique employed, which might represent a potential shift in corporate control" (FAQ). 
This means that under Section 13(d), anyone who becomes the beneficial owner of 5\% of an issuer's equity securities registered under Section 12 of the Exchange Act must file with the SEC a Schedule 13(D) within 10 days after the acquisition -- it informs investors about individuals who could influence or change control of the issuing company \citep{Giglia2018}. Whereas filing a Schedule 13(D) allows the investor to behave in an active manner, a passive investor can file a Schedule 13(G) in lieu of a Schedule 13(D). It is a short-form filing that can be utilized if an investor holds a beneficial ownership interest pasively, with no intent to change control of the company \citep{Giglia2018}. Within the Schedule 13(D) filings is information important to the following analysis. This inculdes the (1) security and the issuer, (2) the identity and background of the filer, (3) the source and amount of funds or other conssiderations and most importantly, (4) the purpose of the transaction. Concluding, a Schedule 13(D) filing contains all information relevant to assess the underlying acquisition of at least 5\% of outstanding stock. The filers can be broadly classified into institutional investors (e.g. hedge funds, mututal funds) other entrepreneurial activists (e.g. individuals) \citep{Klein2009} and corporations. 

\subsection{Hedge Fund Activism}
%Topics: Characteristics AR | Problems in Comparison | Motivation of HF | Activism
There have been many studies that examined the effect a Schedule 13(D) filing submitted by hedge funds has on the target firm's stock price. In the presence of short-hprizon event studies of stock returns they all find positive abnormal returns for the subjects stock around the filing date. 
\citet[p.1730]{Brav2008} find positive average abnormal returns in the range of "7\% to 8\% during the (-20,+20) announcement window for activist hedge funds. \citet{Klein2009} have similar findings and observe 10.2\% average abnormal stock returns for hedge fund targets. In contrast, \citet{Greenwood2009} observe average abnormal announcement returns of 2.36\% for a sample of activist portfolio investors. In a more recent study by \citet{Denes2017}, they average the valuation effect to around 5\% on the target's stock if submitted by hedge funds. It can be seen that all studies observe positive abnormal returns around the filing date but differ in their magnitude (Comparing the the returns can be misleading as the authors used different models for computing the abnormal returns)
	\footnote{\citet{Greenwood2009} use the market return model with matching portfolios and the CAR for aggregated abnormal returns; \citet{Brav2008} calculates the aggregated abnormal returns by subtracting the value-weighted market index from the buy-and-hold return; \citet{Klein2009} use a similar approach with buy-and-hold returns but make more adjustments.}.
Although the filing of a Schedule 13(D) can be seen as the trigger for the market reaction, the reason of why the abnormal returns occur is still largely unknown. \citet[p.12]{Brav2009} however list the main objectives of hedge fund activism based on filings of their sample. The vast majority of these objectives focuses on general characteristics of target and a possible increase in shareholder value. To achieve these goals, hedge fund's objectives can be separated into five, not mutually exclusive motives. The first objective is the believe of the hedge fund that he can help the manager maximize the shareholder value because they believe that the company is undervalued. The second includes activism that is based on the targeting firm's payout policy and capital structure. for the third objective, the hedge funds target issues related to business strategy, such as operational efficiency, mergers and acquisitions or growth strategies. The fourth objective is aimed at the sale of the target company with the majority to force a sale of the target company to a third party. The last objective includes activism targeting corporate governance. These motives are congruent with the \citet{Klein2009} definition of an entrepreneurial activist "who buys a large stake in a publicly held corporation with the intention to bring about change and thereby realize a profit on the investment". A more cautious definition is presented by \citet{Greenwood2009} who define an activist investor as someone who tries to change the status quo through voice, without a change in control of the firm. 

\subsection{Minority Acquisitions}
%Topics: General Objectives | Definitions | Motivation | Minority Acquisitions | Toehold 

While the objectives of hedge funds in the light of Schedule 13(D) filings  have been discussed in many studies, there is still much more debate on the motivation of corporations to make active minority acquisitions. These investments in other firms' equities are sometimes for the general purpose of investment but far more importantly as a strategic interest in the opposing company. As \citet{Allen2000} note, block ownership by corporations is unique relative to block ownership by institutions or individuals because of the possibility that business agreements, alliances, or joint venture might be reached between target firms and corporate owners
	\footnote{In \citet{Allen2000}sample, the mean fraction of equity acquired in the sample is 20\% with blockholdings of at least 5\% of voting shares. This is similar to underlying sample of this paper.}. 
In the context of relationship-specific investments, \citet{Ouimet2013} finds that minority acquisitions can be an incentive for a filer concerned with the holdup problem to enter into the agreement. The integrated ownership helps to remove the threat of an opportunistic renegotiation by the target. The minority acquisition can help to mitigate incomplete contracts and thereby facilitate cooperation between two independent firms \citep{Allen2000}. He classifies miniority acquisitions as acquisitions if less than 50\% acquired, majority acquisitions if above. Another motive of minority acquisitions \citet{Ouimet2013} is the direct financing of the target by the acquirer. Minority acquisitions help to overcome the asymmetric information and help to certify the target for other outside investors. Following \citet{Ouimet2013}, minority acquisitions can also help to assess real options. The corporate investor acquires a minority acquisition in order better assess the target for a potential majority acquisition and to gather more information before launching a bid for full control \citep{Huang2017}. 
Concluding and following \citet{Allen2000},\citet{Ouimet2013} and \citet{Huang2017} corporations make minority acquisitions in other companies when they confront informational or integration barriers. On the other hand, minority acquisitions can help as a stepping stone towards full acquisition and \citet{Huang2017} show that those minority stakes can affect takeover deals that involve greater information asymmetry. "In the United States, there are essentially two ways to acquire a publicly traded firm, either through a merger or through a tender offer. In a merger, the acquirer and the target’s board of directors agree on a price, and then the target’s shareholders vote to approve the deal. In a tender offer, the acquirer proposes a per-share price to the target’s shareholders, and then the shareholders have the choice to sell their shares at the offer price or keep them" \citep[p. 2]{Offenberg2015}. Tender offers used to be characterized as hostile takeovers but are now also considered as friendly takeovers (takeover=merger but it does not have to be mutual). In this context, a toehold is defined as the purchase of an ownership prior to initiating a tender offer. However, the edge of ownership between toeholds and minority stakes is thin. \citet{Eckbo2009} state that acquiring a toehold before initiating a takeover bid is compelling. It reduces the number shares that must be bought at the full takeover premium and it can be sold at an even greater premium should a rival bidder enter the contest and wind the target. 
	\footnote{was für percentages gibt es}










\pagebreak
With regards to stated objective, this paper is combining literature on investor activism and fundamental analysis in determining a companies strength.

% Literature investor activism
Recent research in the field of investor activism by \citet{Brav2008} shows that hedge fund activism has a positive effect on the performance of the target company, creates a favorable market reaction and activist hedge funds have a high succession rate in achieving their main objectives \footnote{They analyse the following objective of activist campaigns: (1) Maximize shareholder value (2) changes in the capital structure (3) changes in the business strategy (4) sale of the target company (5) changes in corporate governance}.
\citet{Klein2009} not only analyse activism by hedge funds but also incorporate private investors into their analysis. In accordance with \citet{Brav2008} they observe a positive market reaction around the announcement date and highlight the success rate of activists in achieving their campaign's main objectives.
\citet{CoffeeJr.2014} are in line with a market runup in response to investor activism by hedge funds but focus on their real value creation. They find that hedge fund activism may result in a severe externalities namely at the shortening of investment horizons and the discouragement of research and development. \citet{Greenwood2009} also document large positive abnormal returns when hedge funds announce their activist intentions and show that the ability to force the target into a takeover is attributable to the abnormal returns. In addition they find that the highest impact on the market is for those ultimately acquired.
While all of these studies involve a deepened investigation of hedge-funds, especially their impact and motivation, most of them leave the remaining investor types aside. In particular, there has been no study that independently evaluates corporate activism and directly investigates the relation of the investor's strength with the subsequent market reaction. 

% Literature financial strength 
 In a study of 2010 \emph{BCG} notes that many of the year's acquisitions would involve a financially strong acquirer. However, the attribute of being financially strong is not ambivalent in its definition. With the objective of separating strong from weak value firms, \citet{Piotroski2000} established the F-score. The F-score represents a simple application of fundamental analysis and is the sum of nine binary signals that form a "... composite measure of firm strength" \citep[p. 496]{Fama2006}. In order to legitimize the explanatory power of the F-score in separating strong from weak firms he formed portfolios.In doing so he showed that an investment strategy of shorting expected losers (weak firms) and buying expected winners (strong firms) would "generate a 23\% average annual return" \citep[p. 4]{Piotroski2000}. \citet{Hyde2014} have matching results and observe significant return premiums for stock with a high F-score over stocks with a low F-score. Although the F-score was established to distinguish among value firms, \citet{Mohr2012} shows that an application on growth stocks yields similar results without loosing the predictive ability \footnote{This is in line with \citet{Piotroski2000} and confirms earlier research conducted by him}.

 In conducting the analysis, the F-score will be used to separate  the sample of 13D filings among strong and weak corporate investors. Since is is able to separate firms in portfolios into strong and weak performing ones, an application to this analysis seems reasonable. 
 
 %insert negative aspects here - accruals 
 However, components of the f-score include changes in leverage and The score itself can be divided into the three dimensions profitability, balance sheet health and operating efficiency. 
 In the context of this analysis As \citet{Mohr2012} states: the f-score considers in what direction the fundamentals of a company are trending and whether financial health conditions are met.  Because high F-scores imply higher returns hence stronger firms should have higher returns, investors must see a high F-score as a representation of financial strength. In the context of this paper those practices would have only been applied to the target and not the investor. An application of the F-score on the investor with the aim of distinguishing between strong and weak firms 


\citet{Choi2012} formulate it from a target perspective - "does financial strength predict subsequent institutional demand"? 

%In general, analyzed characteristics across the sample of filings but not the characteristics of the parties in general.

On the other hand, \citet{Akhigbe2007} examine the characteristics of final acquisitions following partial bids. They find that involvements by corporate bidders are more likely to result in a full acquisition. 

% previous studies have examined the content of the filings and therefre indirectly involved the investor in their analysis and when actively involving a party only analysed the target (s. above paragraph).
% difference of the paper 

\section{Data}

\subsection{Constructing the Sample}
% Why Schedule 13(D)filings - Klein 
The data used to conduct the following analysis is primarily composed of information gathered from Schedule 13(D) filings 
	\footnote{Schedule 13(D) filings are "the mandatory federal securitites law filings under Section 13(d) of the 1934 Exchange Act that investors must file with the SEC within 10 days of acquiring more than 5\% of any class of securities of a publicly traded company if they have an interest in influencing the management of the company" \citep[p. 1736]{Brav2008}} 
within SEC's Edgar database and further from data provided by Wharton Research Data Services (WRDS). The sample of Schedule 13(D) filings is conctructed as follows. First, using an automatic search script, 48'626 filings from the 20 year period starting in January 1996 and ending in December 2016 were identified.  The script identifies all Schedule 13(D) filings that appear on EDGAR and extracts the following information: name of filer and subject, the CUSIP of the underlying security and the filing date. Next, to only have filings submitted by corporations hence to separate corporate investors from institutional investors (i.e. hedge-funds, pension-funds or real estate investment trusts (REITs), 10-K reports were cross-referenced with the initial sample of all filings
	\footnote{10-K reports were used to identify corporations because "managers of publicly traded firms are required to produce public documents that provide a comprehensive review of the firm’s business operations and financial condition and an important financial disclosure document created by managers to communicate with investors and analysts is the annual report filed pursuant to the Securities Exchange Act of 1934 the Form 10-K." \citep[p. 1643]{Loughran2014}}. 
In order to be part of the sample, the filer had to have a 10-K report submitted 12 months prior to the filing which reduced the sample to 3'325 filings. Because the daily stock returns and prices for the underlying securities come from the Center for Research in Security Prices (CRSP) the subject not only had to have SEC's CUSIP identifier but also an active link between its CUSIP and CRSP's unique PERMNO identifier. For the remaining 1'467 filings, there had to be sufficient data on CRSP in order to calculate the abnormal returns for the subjects which reduced the sample to 1'151 filings. 
The accounting fundamentals, needed to compute the filers financial condition, come from the COMPUSTAT database which means that the filer has to have a link between its 10K-CIK and COMPUSTAT's unique GVKEY indentifier. After crossreferencing with the remaining 1'151 filings, the sample was reduced to 1'014 filings. In the next step, according to Fama \& French's industry classification code, all filers belonging to the trading industry (Code 47) were dropped which left a sample size of 898 filings. In a last step, size and purpose of the transaction were manually extracted from the Schedule 13(D) filings, while in the process Schedule 13(D/A) filings (e.g. amendements to previous filings) that were mistakenly classified as original Schedule 13(D) filings and filings not submitted by corporations were excluded.

\subsection{Descriptive Data}
\subsection{Examples out of the Sample}

\end{document}