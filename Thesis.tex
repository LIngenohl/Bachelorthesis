\documentclass[12pt]{article}
	\usepackage[T1]{fontenc}
	\usepackage[utf8]{inputenc}
	\usepackage[british]{babel}
	\usepackage[a4paper]{geometry}
	\geometry{verbose,tmargin=3cm,bmargin=3cm,lmargin=2cm,rmargin=2cm,marginparwidth=70pt}
	\setcounter{secnumdepth}{3}
	\setcounter{tocdepth}{3}
	\setlength{\parindent}{4em}
	\setlength{\parskip}{1em}
	\renewcommand{\baselinestretch}{1.5}
	\usepackage{prettyref}
	\usepackage{textcomp}
	\usepackage{setspace}
	\usepackage{indentfirst}
	\usepackage{fancyhdr}
	\usepackage{url}
	\usepackage[normalem]{ulem}
	\usepackage[table, fixpdftex]{xcolor}
	\usepackage{algpseudocode}
	\usepackage{bigstrut}
	\usepackage{enumitem}

	% package hyperref
    \usepackage[hidelinks]{hyperref}
  

	% biblatex
	\usepackage[style=authoryear,natbib=true,maxcitenames=2, maxbibnames=11,backend=biber,pagetracker=page,hyperref=true]{biblatex} \usepackage{csquotes}
	\renewcommand*{\bibsetup}{%
		\interlinepenalty=10000\relax % default is 5000
		\widowpenalty=10000\relax
		\clubpenalty=10000\relax
		\raggedbottom
		\frenchspacing
        \biburlsetup}
        
	% fixes the page number of the first page of each chapter
	\fancypagestyle{plain}{
			\fancyhead{}
			\renewcommand{\headrulewidth}{0pt}
			\renewcommand{\footrulewidth}{0pt}
			\fancyfoot[OC]{\begin{flushright}\thepage\end{flushright}}
    }
    
	% fancy headers for the thesis
	\fancyhead{}
	\fancyhead[RO]{\slshape \nouppercase \rightmark}
	\fancyfoot[OC]{\begin{flushright}\thepage\end{flushright}}
	\renewcommand{\headrulewidth}{0.4pt}
	\setlength{\headheight}{14pt}

	% add bibliography database
	\addbibresource{BA Kopie.bib}
	
	% space between biblio items
	\setlength\bibitemsep{1.7\itemsep} 
	
	% title without ""
	\DeclareFieldFormat[inbook]{title}{#1}
	% non-italic
	\DeclareFieldFormat[online]{tlaitle}{#1} 
	% title unquoted
	\DeclareFieldFormat[article]{title}{#1} 
	% no pp. 
	\DeclareFieldFormat[article]{pages}{#1} 
	% bold volume
	\DeclareFieldFormat*{volume}{\mkbibbold{#1}\setpunctfont{\textbf}}
	
	% no in:
	\renewbibmacro{in:}{} 
	
	% (volume)
	\renewbibmacro*{volume+number+eid}{%
			\printfield{volume}%
			%\setunit*{\adddot}% DELETED
			% \setunit*{\addnbspace}% NEW (optional); there's also \addnbthinspace
			\printfield{number}%
			% \setunit{\addcomma\space}%
			\printfield{eid}}
	\DeclareFieldFormat[article]{number}{\mkbibparens{#1}} 
	
	% edition.
	\DeclareFieldFormat{edition}%
	{(\ifinteger{#1}%
			{\mkbibordedition{#1}\addthinspace{}ed.}%
			{#1\isdot}).}
	
	% publisher and location position
	\renewbibmacro*{publisher+location+date}{%
			\printlist{publisher}%
			\setunit*{\addcomma\space}%
			\printlist{location}%
			\setunit*{\addcomma\space}%
			\usebibmacro{date}%
			\newunit}
	
	% shortauthor before author
	\renewbibmacro*{begentry}{%
			\ifkeyword{Key}{\sffamily}{}%
			\iffieldundef{shorthand}
			{}
			{\global\undef\bbx@lasthash
					\printfield{shorthand}%
					\addcolon\space}%
			\ifboolexpr{test {\usebibmacro{bbx:dashcheck}} or test {\ifnameundef{shortauthor}}}%
			{}%
			{\printnames{shortauthor}%
                    \addspace\textendash\space}}
                    
\title{Thesis}
\author{Leopold Ingenohl}


\begin{document}
\maketitle

Much attention has been recently given to the current Securities and Exchange Commission reporting requirements for Schedule 13D, governing the disclosure of beneficial ownership interests in excess of five percent of outstanding common stock of a U.S. public company \citep{Giglia2018}. Amongst other causes, it is due to the fact that the targeted corporation experiences significant gains when the partial acquisition is announced \citep{Akhigbe2007}. % insert anecdotal reference here - such filings... 
% insert result of argumentation here

In fact, many event studies have been conducted for analysis of what happens to the target's stock when there is such a filing. However, it is still largely unanswered where this upward drift comes from \citep{Greenwood2009}. An approach to this issue was presented by \citet{Akhigbe2007}, mentioning that the market reaction is associated with the size of the announced partial position and the degree of the target's free cash-flow.
\citet{Greenwood2009} address this issue by presuming that the runup is a reflection of investors' expectations of the target firm being acquired at a premium to the current stock price.\\ 
Another approach is objective of this paper. It's aim is to examine the link between the company condition of both corporate investor and target and the subsequent market reaction on announcement of the filing. %This approach is somewhat limited because institutional investors do not have a balance sheet.
The following findings motivate this approach.  
% insert 
In recent studies of what happens to the target's stock  \citet{Collin-Dufresne2015} observed a positive significant market reaction upon a more general sample of Schedule 13D filings inculding all investor types. \citet{Brav2008} have shown a favorable market reaction -- 7\%-8\% average abnormal returns in the (-20|20) event window -- particularly to Schedule 13D's filed by hedge funds. Similar results have been shown by \citet{Klein2009} who observe 10.2\% average abnormal stock returns specifically for hedge fund targets. Furthermore the runup is even higher if the acquirer is a private investor or a non-financial corporation \citep{Brigida2012}. This is matching with \citet{Akhigbe2007} findings who observed greater gains for the target's stock if the partial position was initiated by a corporate bidder. Concluding, all filings are followed by positive market reactions but those submitted by corporations seem to have a stronger impact. 

Beyond the effect these filings have on the target's stock, \citet{Akhigbe2007} observe that partial acquisitions, if carried out by corporate investors, are more likely to result in a full acquisition when compared to all other activist investors. Within the mass of Schedule 13D filings, institutional investors are unlikely to pursue a complete takeover whereas corporations are potential full acquirers \citep{Brigida2012}. This would indicate a higher probability for corporate investors to ultimately acquire the target following their minority stake in it \citep{Greenwood2009} \footnote{expressed by a SC 13D filing}. 

Schedule 13D filings by corporate investors, compared to other investor types, seem to have a stronger impact on the targets stock. Considering this, a limitation of the underlying sample to only corporate investors could be of reasonable interest. 
By giving these filings a higher probability of acquisition and tracing the runups back to this assumption, the financial condition of the investor -- in order to carry out a possible acquisition -- should play an important role. For the reason that a strong investor could increase the likelihood of takeover and hence explain the strong abnormal returns.
%not least because institutional investors would not have similar characteristics for this evaluation . 
 
Based on these findings, the economic significance of corporate cross-holdings in the context of investor activism is apparent and the link between the financial condition of the investor and the subsequent abnormal returns on the target stock is an interesting issue to examine. 

\section{Literature Review/Theory}

With regards to stated objective, this paper is combining literature on investor activism and fundamental analysis in determining a companies strength.

% Literature investor activism
Recent research in the field of investor activism by \citet{Brav2008} shows that hedge fund activism has a positive effect on the performance of the target company, creates a favorable market reaction and activist hedge funds have a high succession rate in achieving their main objectives \footnote{They analyse the following objective of activist campaigns: (1) Maximize shareholder value (2) changes in the capital structure (3) changes in the business strategy (4) sale of the target company (5) changes in corporate governance}.
\citet{Klein2009} not only analyse activism by hedge funds but also incorporate private investors into their analysis. In accordance with \citet{Brav2008} they observe a positive market reaction around the announcement date and highlight the success rate of activists in achieving their campaign's main objectives.
\citet{CoffeeJr.2014} are in line with a market runup in response to investor activism by hedge funds but focus on their real value creation. They find that hedge fund activism may result in a severe externalities namely at the shortening of investment horizons and the discouragement of research and development. \citet{Greenwood2009} also document large positive abnormal returns when hedge funds announce their activist intentions and show that the ability to force the target into a takeover is attributable to the abnormal returns. In addition they find that the highest impact on the market is for those ultimately acquired.
While all of these studies involve a deepened investigation of hedge-funds, especially their impact and motivation, most of them leave the remaining investor types aside. In particular, there has been no study that independently evaluates corporate activism and directly investigates the relation of the investor's strength with the subsequent market reaction. 

% Literature financial strength 
 In a study of 2010 \emph{BCG} notes that many of the year's acquisitions would involve a financially strong acquirer. However, the attribute of being financially strong is not ambivalent in its definition. With the objective of separating strong from weak value firms, \citet{Piotroski2000} established the F-score. The F-score represents a simple application of fundamental analysis and is the sum of nine binary signals that form a "... composite measure of firm strength" \citep[p. 496]{Fama2006}. In order to legitimize the explanatory power of the F-score in separating strong from weak firms he formed portfolios.In doing so he showed that an investment strategy of shorting expected losers (weak firms) and buying expected winners (strong firms) would "generate a 23\% average annual return" \citep[p. 4]{Piotroski2000}. \citet{Hyde2014} have matching results and observe significant return premiums for stock with a high F-score over stocks with a low F-score. Although the F-score was established to distinguish among value firms, \citet{Mohr2012} shows that an application on growth stocks yields similar results without loosing the predictive ability \footnote{This is in line with \citet{Piotroski2000} and confirms earlier research conducted by him}.

 In conducting the analysis, the F-score will be used to separate  the sample of 13D filings among strong and weak corporate investors. Since is is able to separate firms in portfolios into strong and weak performing ones, an application to this analysis seems reasonable. 
 
 %insert negative aspects here - accruals 
 However, components of the f-score include changes in leverage and The score itself can be divided into the three dimensions profitability, balance sheet health and operating efficiency. 
 In the context of this analysis As \citet{Mohr2012} states: the f-score considers in what direction the fundamentals of a company are trending and whether financial health conditions are met.  Because high F-scores imply higher returns hence stronger firms should have higher returns, investors must see a high F-score as a representation of financial strength. In the context of this paper those practices would have only been applied to the target and not the investor. An application of the F-score on the investor with the aim of distinguishing between strong and weak firms 


\citet{Choi2012} formulate it from a target perspective - "does financial strength predict subsequent institutional demand"? 

%In general, analyzed characteristics across the sample of filings but not the characteristics of the parties in general.

On the other hand, \citet{Akhigbe2007} examine the characteristics of final acquisitions following partial bids. They find that involvements by corporate bidders are more likely to result in a full acquisition. 

% previous studies have examined the content of the filings and therefre indirectly involved the investor in their analysis and when actively involving a party only analysed the target (s. above paragraph).
% difference of the paper 

\section{Overview}
% 13D filings
% F-Score
% Expected Return 
\section{Data}
\subsection{Contructing the Sample}
Basis of the following analysis are all Schedule 13D filings from the 13 year period starting in 2004 and ending blabla 2017 (available on SEC's EDGAR).
Information important to the analysis and contained in the filings are (1) the filing date, (2) the filer and (3) the subject. The sample is then constructed as follows. As the analysis is focusing on corporate investors and their cross-holdings the sample is further restricted by cross-referencing the 13D filings with a corresponding sample of 10-K filings of the firms. It is done to separate corporate investors from institutional investors (e.g. hedge-funds, pension-funds,real estate investment trusts (REITs)). 
The daily stock returns and prices come from the Center for Research in Security Prices (CRSP). The accounting fundamentals come from the COMPUSTAT database. This is in accordance with \citet{Fama2006} and \citet{Brigida2012} resulting in (1) an exclusion of targets in the financial and utility industries, (2) firms being required to be listed CRSP and have share codes (10) or (11) (i.e. ordinary shares) and (3)  firms being listed in the COMPUSTAT database and having sufficient data for the relevant periods. In a last step company web sites and newspaper articles are used to determine whether or not the reamaining activists are corporate investors or not.\\
This sample differs from \citet{Brav2008} who considered only filings by hedge funds, and \citet{Collin-Dufresne2015} who involved all investor types in their sample. Table \emph{xx} summarizes the selection process. % describe table here
% insert outlook here


\end{document}