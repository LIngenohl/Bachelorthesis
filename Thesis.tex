\documentclass[12pt]{article}
	\usepackage[T1]{fontenc}
	\usepackage[utf8]{inputenc}
	\usepackage[british]{babel}
	\usepackage[a4paper]{geometry}
	\geometry{verbose,tmargin=3cm,bmargin=3.5cm,lmargin=4cm,rmargin=3cm,marginparwidth=70pt}
	\setcounter{secnumdepth}{3}
	\setcounter{tocdepth}{3}
	\setlength{\parindent}{4em}
	\setlength{\parskip}{1em}
	\renewcommand{\baselinestretch}{1.5}
	\usepackage{prettyref}
	\usepackage{textcomp}
	\usepackage{setspace}
	\usepackage{indentfirst}
	\usepackage{fancyhdr}
	\usepackage{url}
	\usepackage[normalem]{ulem}
	\usepackage[table, fixpdftex]{xcolor}
	\usepackage{algpseudocode}
	\usepackage{bigstrut}
	\usepackage{enumitem}

	% package hyperref
    \usepackage[hidelinks]{hyperref}
  

	% biblatex
	\usepackage[style=authoryear,natbib=true,maxcitenames=2, maxbibnames=11,backend=biber,pagetracker=page,hyperref=true]{biblatex} \usepackage{csquotes}
	\renewcommand*{\bibsetup}{%
		\interlinepenalty=10000\relax % default is 5000
		\widowpenalty=10000\relax
		\clubpenalty=10000\relax
		\raggedbottom
		\frenchspacing
        \biburlsetup}
        
	% fixes the page number of the first page of each chapter
	\fancypagestyle{plain}{
			\fancyhead{}
			\renewcommand{\headrulewidth}{0pt}
			\renewcommand{\footrulewidth}{0pt}
			\fancyfoot[OC]{\begin{flushright}\thepage\end{flushright}}
    }
    
	% fancy headers for the thesis
	\fancyhead{}
	\fancyhead[RO]{\slshape \nouppercase \rightmark}
	\fancyfoot[OC]{\begin{flushright}\thepage\end{flushright}}
	\renewcommand{\headrulewidth}{0.4pt}
	\setlength{\headheight}{14pt}

	% add bibliography database
	\addbibresource{BA Kopie.bib}
	
	% space between biblio items
	\setlength\bibitemsep{1.7\itemsep} 
	
	% title without ""
	\DeclareFieldFormat[inbook]{title}{#1}
	% non-italic
	\DeclareFieldFormat[online]{tlaitle}{#1} 
	% title unquoted
	\DeclareFieldFormat[article]{title}{#1} 
	% no pp. 
	\DeclareFieldFormat[article]{pages}{#1} 
	% bold volume
	\DeclareFieldFormat*{volume}{\mkbibbold{#1}\setpunctfont{\textbf}}
	
	% no in:
	\renewbibmacro{in:}{} 
	
	% (volume)
	\renewbibmacro*{volume+number+eid}{%
			\printfield{volume}%
			%\setunit*{\adddot}% DELETED
			% \setunit*{\addnbspace}% NEW (optional); there's also \addnbthinspace
			\printfield{number}%
			% \setunit{\addcomma\space}%
			\printfield{eid}}
	\DeclareFieldFormat[article]{number}{\mkbibparens{#1}} 
	
	% edition.
	\DeclareFieldFormat{edition}%
	{(\ifinteger{#1}%
			{\mkbibordedition{#1}\addthinspace{}ed.}%
			{#1\isdot}).}
	
	% publisher and location position
	\renewbibmacro*{publisher+location+date}{%
			\printlist{publisher}%
			\setunit*{\addcomma\space}%
			\printlist{location}%
			\setunit*{\addcomma\space}%
			\usebibmacro{date}%
			\newunit}
	
	% shortauthor before author
	\renewbibmacro*{begentry}{%
			\ifkeyword{Key}{\sffamily}{}%
			\iffieldundef{shorthand}
			{}
			{\global\undef\bbx@lasthash
					\printfield{shorthand}%
					\addcolon\space}%
			\ifboolexpr{test {\usebibmacro{bbx:dashcheck}} or test {\ifnameundef{shortauthor}}}%
			{}%
			{\printnames{shortauthor}%
                    \addspace\textendash\space}}
                    
\title{Thesis}
\author{Leopold Ingenohl}


\begin{document}
\maketitle

\section{Introduction}
Much attention has been recently given to the current Securities and Exchange Commission reporting requirements for Schedule 13D, governing the disclosure of beneficial ownership interests in excess of five percent of outstanding common stock of a U.S. public company \citep{Giglia2018}. Amongst other causes, it is due to the fact that the targeted corporation experiences significant gains when the partial acquisition is announced \citep{Akhigbe2007}. % insert anecdotal reference here - such filings... 

In fact, many event studies have been conducted for analysis of what happens to the target's stock when there is such a filing. Whereas \citet{Collin-Dufresne2015} observed a positive significant market reaction upon a more general sample of Schedule 13D filings, \citet{Brav2008} have shown a favorable market reaction -- 7\%-8\% average abnormal returns in the (-20|20) event window -- particularly to Schedule 13D's filed by hedge funds during the period surrounding the initial Schedule 13D. Similar results have been shown by \citet{Klein2009} who observe 10.2\% average abnormal stock returns specifically for hedge fund targets. With regards to the samples of general and hedge-fund filings, the runup is even higher if the acquirer is a private investor or a non-financial corporation \citep{Brigida2012}. This is matching with \citet{Akhigbe2007} findings who observed greater gains for the target's stock if the partial position was initiated by a corporate bidder. 

Beyond the effect these filings have on the target's stock, \citet{Akhigbe2007} observe that partial acquisitions, if carried out by corporate investors, are more likely to result in a full acquisition when compared to all other activist investors. This is equivalent to the findings that within the mass of Schedule 13D filings, institutional investors are unlikely to pursue a complete takeover whereas corporations are potential full acquirers \citep{Brigida2012}. Hence following their minority stake (expressed by a SC 13D filing) in the target, corporate bidders have a higher probability of ultimately acquiring it \citep{Greenwood2009}. But in order to carry out a full acquisition, corporate investors need to be in the state of sufficient financial strength. 
% bridge between full acquisition and financial condition of the investor - takeover! 
Although there is not much debate on whether schedule 13D filings elicit a market reaction, it is still largely unanswered where this upward drift in the target's stock price comes from \citep{Greenwood2009}. \citet{Greenwood2009} address this issue by presuming that the runup is a reflection of investors' expectations of the target firm being acquired at a premium to the current stock price. Another approach to address this issue is presented by \citet{Akhigbe2007}, mentioning that the market reaction is associated with the size of the announced partial position and the degree of the target's free cash-flow. 

Because filings submitted by corporate investors, compared to all other investors, are followed by larger average abnormal returns, it seems  out of the pool of filings they have the highest impact. If additionally, these filings are characterized by a higher probability of full acquisition which is dependent on the investors financial strength, attention of analysis in understanding the upward drift should be channelled towards the financial condition of the corporate investor. 
% previous studies have examined the content of the filings and therefore indirectly involved the investor in their analysis and when actively involving a party only analysed the target (s. above paragraph).
Based on these studies, the economic significance of corporate cross-holdings in the context of investor activism is apparent and the link between the financial condition of the investor and the subsequent abnormal returns on the target stock is an important issue to examine. 

\end{document}